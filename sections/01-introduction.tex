\section{Introduction}

%\lipsum[8]

%\lipsum[8]

% scientific Issue

% state of the art -> scientific literature results
% new Hypothesis and required knowledge
% overview of contents of following sections
% clearly state the aims of the study
% relevance of the results

% SAveRUNNER Hypothesis
% drug targets and associated disease genes/modules are nearby in the interactome map
% looking for high similarity values
% the program quantifies interplay between drug targets and disease proteins (# of edges) 

The Coronavirus disease 2019 (COVID-19) pandemic is the third outbreak of coronavirus-related disease of the 21st century. What all three have in common is the unpredicted emergence, rapid and easy proliferation that leads to harsh consequences in each case. However, they do differ in the characteristics of the diseases.\\
The first outbreak, Severe acute respiratory syndrome (SARS), was caused by severe acute respiratory syndrome coronavirus (SARS-CoV-1). The SARS disease first occurred in Foshan, China in November 2002. In 2003 the infections happened globally, leading to a ~10\% fatality rate.
The disease caused by Middle East respiratory syndrome coronavirus (MERS-CoV) first emerged at Jeddah in Saudi Arabia in 2012 and developed a fatality rate of ~35\% globally \cite{Kirtipal_2020}. 
COVID-19, which is caused by SARS-CoV-2, has a much lower fatality rate which is below 3\% in most countries \cite{Mathieu2021, Yi_2020}. It is important to note that the former two outbreaks have much lower counted infections than COVID-19. In summary, COVID-19 spreads much faster and more successful, but leads to fewer deaths among the infected. This makes the disease harder to control and allowed the outbreak to reach a scale and severity not known before from Covid-19's predecessors. 
In addition, there appears to be a similarity in symptoms as well. The main early symptoms, fever, dry cough, dyspnea, and diarrhea occurred in similar percentages among the infected for all three diseases. However, MERS patients required ventilation support in 80\%  of cases. For SARS and COVID-19, this number is less than 20\% and only 8\% respectively \cite{Yi_2020}.
Further symptoms of COVID-19 include shortness of breath, muscle ache, dizziness, headache, sore throat, rhinorrhea, chest pain, diarrhea, nausea, vomiting, hypoxemia, acute respiratory syndrome, septic shock, metabolic acidosis, and coagulopathy. 
Furthermore, atypical pneumonia, acute lung injury, and acute respiratory distress syndrome (ARDS) are associated with the worst chest radiographic findings in patients \cite{Yi_2020}.
In addition, all three viruses are believed to originate from bats, which are believed to be hosts of emerging viral pathogens due to their unique immune systems, their densely packed colonies, longevity, and ability to fly which allows spread at a fast rate. Bats act as host to other coronaviruses as well \cite{Kirtipal_2020}.
The intermediate hosts for SARS, MERS, and COVID-19 are believed to be civet cats, camels, and pangolins respectively. The latter is based on a 99\% similarity of SARS-CoV-2 and another coronavirus found in the pangolin \cite{Yi_2020}. This information is crucial for cutting the transmission line to prevent re-emergence in the future. 
Infection takes place via person‐to‐person interactions, transmission by aerosols, and transmission by touch. The transmission is also influenced by several environmental factors. For instance, when the temperature is high, the rate of transmission is inhibited. impact of maximum air humidity and wind speed on the prevalence of COVID‐19 has been found not to be statistically significant \cite{Sreekanth_Reddy_2020}. \\

The virus body of SARS-CoV-2 consists of a single-stranded RNA genome and a protein hull surrounding it. The viral spike glycoprotein of SARS-CoV-2 was found to interact with Angiotensin-converting enzyme 2 (ACE2) on the cells membrane to trigger membrane fusion of the virus and its host. Thus all cells that express ACE2 (lung, kidney \& gastrointestinal system) are targets of SARS-CoV-2. Despite that respiratory failure remains the primary cause of death \cite{Yi_2020}.
Without going into too much detail about the human immune response, it is important to note that later stages of COVID-19 are characterized by a cytokine storm causing hyper-inflammation. Furthermore, SARS-CoV-2 possesses the ability to halt T cell function and to induce their apoptosis as well as causing pyroptosis in lymphocytes and macrophages \cite{Kirtipal_2020}.
Overall, the genome of corona-viruses shows high flexibility in terms of gene content and recombination. In one investigation, the spike protein was found to have the most variation in
its active sites while other structural proteins (eg. the E protein) were highly conserved \cite{Ahmadi_2021}. This raises the question of whether other proteins are better suited for vaccines than the currently used spike protein. \\

There are two main groups of drugs that are currently considered for the treatment of COVID-19. Those are antiviral \& immunomodulatory drugs. There are other candidates as well, which cannot be grouped as easily \cite{Bartoli_2021}. Favipiravir (FPV) and Remdesivir (RDV) appear to be the most promising antiviral drugs but many more have been suggested \cite{Yi_2020, Mohamed_2021, Sreekanth_Reddy_2020}. PFV is a type of RNA-dependent RNA polymerase (RdRp) inhibitor. It is also known as T-705 and was being developed in 2002 as an inhibitor of influenza. It selectively inhibits the viral RNA-dependent RNA polymerase or causes lethal mutagenesis upon incorporation into the virus RNA without cytotoxicity to mammalian cells \cite{Ghasemnejad_Berenji_2020}. This repurposable drug has already been shown to be effective against other single-stranded RNA viruses as well \cite{Ghasemnejad_Berenji_2020, Sreekanth_Reddy_2020}. In addition, it already has been tested in clinical trials in China and has proven to be effective in reducing viral replication. However, the drug has many adverse side effects that make universal and widespread use in the future difficult. Still, these effects are significantly fewer than in some other candidates, such as lopinavir and ritonavir \cite{Ghasemnejad_Berenji_2020}.
Meanwhile, RDV appears to be the most effective antiviral drug proposed so far \cite{Bartoli_2021}. This drug also interacts with the RdRp leading to premature termination of viral RNA replication \cite{Sreekanth_Reddy_2020}. At least in one study, this drug showed no direct negative effects in the patient \cite{Holshue_2020}. It is also the first agent recommended for authorization in the European Union (EU) for use in the treatment of COIVID‐19 \cite{Sreekanth_Reddy_2020}.
Both, FPV and RDV, have also shown little drug-drug interaction with antipsychotic drugs, making them safer to use in the affected patient group \cite{Plasencia_Garc_a_2021}.
Immunomodulatory include corticosteroids like dexamethasone and other drugs, which might be useful in the inflammatory phase of the disease \cite{Bartoli_2021}. Other Inflammation inhibitors, such as anti-IL6, anti-IL1 and are valuable candidates for treatment in advanced stages as well \cite{Stasi_2020, Saeed_2021}.
Azithromycin is another interesting candidate as it combines antiviral and immunomodulatory properties. It has the ability to downregulate cytokine production, maintain epithelial integrity, and prevent lung fibrosis could play a role in the hyperinflammatory stage \cite{Echeverr_a_Esnal_2020}.
Among the other drugs that are suggested are antimalarials like chloroquine and hydroxychloroquine that could have a beneficial effect on COVID-19 through multiple molecular effects \cite{Ong_2020}. In addition to these conventional drugs, efforts on the use of plant-based and traditional Chinese medicine have been made as well, some showing positive results \cite{Yi_2020, Boozari_2020, Adhikari_2020}
\\

Here, we performed network analysis on the human interactome using SAveRUNNER to discover new potential candidates for drug repurposing \cite{Fiscon2020, Fiscon2021}. This process allows finding drugs that have already been approved by the FDA to be used in other diseases.
The benefit of repurposed drugs is less time and lower cost required until the drug can be administered commonly for treatment. As some of the approval steps don't have to be repeated or can be significantly shortened. Although a lot of research has been conducted already, it is still beneficial to cross-check pre-existing results and potentially reveal new candidates to be used for the treatment of COVID-19.


%%%%%%%%% Resterampe

%  10.1002/cbic.202000595 Sreekanth_Reddy_2020
% Coronavirus disease 2019 (COVID‐19) emerges as an infectious disease whose mortality rate is ill‐elucidated because of the difficulty in quantifying the number of asymptomatic and undiagnosed deaths. [1] While some COVID‐19 patients display mild symptoms, others die within a few days after infection.

% https://dx.doi.org/10.1007%2Fs00213-020-05716-4 Plasencia_Garc_a_2021
% The main interactions between COVID-19 drugs and antipsychotics are the risk of QT prolongation and/or TdP, and CYP interactions.
% COVID-19 drugs and antipsychotics are the risk of QT-prolongation and TdP, and cytochromes P450 interactions
% Remdesivir, baricinitib, and anakinra can be used concomitantly with antipsychotics without risk of drug-drug interaction (except for hematological risk with clozapine and baricinitib)
% Favipiravir only needs caution with chlorpromazine and quetiapine.
% Tocilizumab is rather safe to use in combination with antipsychotics. 
% The most demanding COVID-19 treatments for coadministration with antipsychotics are chloroquine, hydroxychloroquine, azithromycin, and lopinavir/ritonavir because of the risk of QT prolongation and TdP and cytochromes interactions

% 10.1007/s12035-020-02093-z Ong_2020
% COVID-19 is a pro-inflammatory-driven condition with loss of smell and taste, suggesting that it may affect the olfactory and gustatory systems and the brain. These effects may persist even after the virus has been cleared from the body

% 10.1016/j.ejphar.2020.173644 Stasi_2020
%  first stage: characterised by upper respiratory tract infection, accompanied by fever, muscle fatigue and pain. Nausea or vomiting and diarrhoea are infrequent in this initial stage of the disease
% second stage - onset of dyspnoea and pneumonia.
%  third stage - worsening clinical scenario dominated by a cytokine storm and the consequent hyperinflammatory state -> leading arterial and venous vasculopathy in the lung with thrombosis of the small vessels and evolution towards serious lung lesions up to ARDS etc.
% fourth stage -> death or recover